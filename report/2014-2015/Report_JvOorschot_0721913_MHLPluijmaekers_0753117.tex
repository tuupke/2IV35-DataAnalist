\documentclass[a4paper,twoside,11pt]{article}
\usepackage{a4wide,graphicx,fancyhdr,amsmath,amssymb}
\usepackage{algorithmic}

%----------------------- Macros and Definitions --------------------------

\setlength\headheight{20pt}
\addtolength\topmargin{-10pt}
\addtolength\footskip{20pt}

\newcommand{\N}{\mathbb{N}}
\newcommand{\ch}{\mathcal{CH}}
\everymath{\displaystyle}
\newcommand{\solution}[1]{\noindent{\bf Solution to Exercise #1:}}

\fancypagestyle{plain}{%
\fancyhf{}
\fancyhead[LO,RE]{\sffamily\bfseries\large Technische universiteit Eindhoven}
\fancyhead[RO,LE]{\sffamily\bfseries\large 2IV35 Visualization}
\fancyfoot[LO,RE]{\sffamily\bfseries\large department of mathematics and computer science}
\fancyfoot[RO,LE]{\sffamily\bfseries\thepage}
\renewcommand{\headrulewidth}{0pt}
\renewcommand{\footrulewidth}{0pt}
}

\pagestyle{fancy}
\fancyhf{}
\fancyhead[RO,LE]{\sffamily\bfseries\large Technische universiteit Eindhoven}
\fancyhead[LO,RE]{\sffamily\bfseries\large 2IV35 Visualization}
\fancyfoot[LO,RE]{\sffamily\bfseries\large department of mathematics and computer science}
\fancyfoot[RO,LE]{\sffamily\bfseries\thepage}
\renewcommand{\headrulewidth}{1pt}
\renewcommand{\footrulewidth}{0pt}

%-------------------------------- Title ----------------------------------

\title{\vspace{-\baselineskip}\sffamily\bfseries 2IV35 Visualization Set 2}
\author{Jeroen van Oorschot \qquad Student number: 0721913 \\{\tt j.v.oorschot@student.tue.nl}\\ \\Mart Pluijmaekers \qquad Student number: 0753117 \\{\tt m.h.l.pluijmaekers@student.tue.nl}}

\date{\today}

%--------------------------------- Text ----------------------------------

\begin{document}
\maketitle

\pagebreak
\tableofcontents
\newpage
\section{Dataset}
Many people in the world take joy in drinking wine although many of them do not have a clue about what they are actually drinking and what makes their wine taste better then others. Many people go to wine-connaisseurs for advice, or enter liquor stores where usually someone has some knowledge about wines, to help give advice on the purchase. However, has many properties which can be measured other than the year. For example, sweetness, acidity and even chlorides play a part in the taste \'and quality of the wine. Not even the best and most experienced connaisseurs can possibly have tasted and formed an opinion on every wine in existance. 


Because I like to a good glass of wine I took the wine dataset to try and find out what makes a wine taste better or worse. The dataset has some measurements of the wine combined with a quality mark determined by a group of experts. There were 2 sets, one with data about red wine, the other about white wine. While looking at the datasets a few things became clear:

\begin{description}
\item[Quality] \hfill \\ The quality is given in an integer number between 0 (very bad) and 10 (very excellent), this means there is not a lot diversity in it.
\item[Dependencies] \hfill \\ Not all of the data is totaly independent, one very clear example are the columns free sulfur dioxide and total sulfur dioxide where the free sulfur dioxide clearly is a part of the total sulfur dioxide.
\item[Red vs white] \hfill \\ In the red wine dataset there is a total of 1599 wines, while in the dataset for white wines there is a total of 4898 wines.
\end{description}


\end{document}
